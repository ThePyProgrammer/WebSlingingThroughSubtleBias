% This is samplepaper.tex, a sample chapter demonstrating the
\documentclass[conference]{IEEEtran}

\usepackage{array}
\usepackage{pgfplotstable}
\usepackage{booktabs}
\usepackage{bbm}      
\usepackage{subfig}
\usepackage{graphicx,dblfloatfix}
\usepackage{caption}
\usepackage{color}
\usepackage{amssymb}   
\usepackage{amsmath}
\usepackage{amsthm}   
\usepackage{algpseudocode}
\usepackage{enumerate}
\usepackage{tabularx}  
\usepackage{multirow}
\usepackage[a4paper=false,
            citecolor=blue,
            colorlinks=true,
            urlcolor=blue,
            linkcolor=blue,
            pdfauthor={Tran Duc Khang},
            pdftitle={Breaking the Web of Prejudice: Exploring Online Racism and Anti-Black Sentiment through Spider-Man},
            pdfsubject={Breaking the Web of Prejudice: Exploring Online Racism and Anti-Black Sentiment through Spider-Man},
            pdfkeywords={white supremacy, anti-black sentiment, Spider-Man, Peter Parker, Miles Morales, comic books, superheroes}
            ]{hyperref}      
\usepackage{url} 
\usepackage[T1]{fontenc}
\usepackage{enumerate}
\usepackage{float}
\usepackage{cite}
\usepackage{verbatim}
\usepackage{listings}\usepackage[left=1.7272cm,right=1.7272cm,top=1.778cm,bottom=2.45cm]{geometry}

\colorlet{punct}{red!60!black}
\definecolor{background}{HTML}{EEEEEE}
\definecolor{delim}{RGB}{20,105,176}
\colorlet{numb}{magenta!60!black}
\newcommand{\mao}[1]{\textcolor{orange}{#1}}

\lstdefinelanguage{json}{
    basicstyle=\scriptsize\ttfamily,
    numbers=right,
    numberstyle=\scriptsize,
    stepnumber=1,
    numbersep=1pt,
    captionpos=b, 
    showstringspaces=false,
    breaklines=true,
    frame=lines,
    backgroundcolor=\color{background},
    literate=
     *{0}{{{\color{numb}0}}}{1}
      {1}{{{\color{numb}1}}}{1}
      {2}{{{\color{numb}2}}}{1}
      {3}{{{\color{numb}3}}}{1}
      {4}{{{\color{numb}4}}}{1}
      {5}{{{\color{numb}5}}}{1}
      {6}{{{\color{numb}6}}}{1}
      {7}{{{\color{numb}7}}}{1}
      {8}{{{\color{numb}8}}}{1}
      {9}{{{\color{numb}9}}}{1}
      {:}{{{\color{punct}{:}}}}{1}
      {,}{{{\color{punct}{,}}}}{1}
      {\{}{{{\color{delim}{\{}}}}{1}
      {\}}{{{\color{delim}{\}}}}}{1}
      {[}{{{\color{delim}{[}}}}{1}
      {]}{{{\color{delim}{]}}}}{1},
}

\begin{document}

% paper title    
% can use linebreaks \\ within to get better formatting as desired

\title{Breaking the Web of Prejudice: Exploring Online Racism and Anti-Black Sentiment through Spider-Man}
% author names and affiliations
% use a multiple column layout for up to three different
% affiliations    
\author{
\IEEEauthorblockN{Tran Duc Khang}
\IEEEauthorblockA{Department of Arts and Humanities \\
NUS High School of Math and Science\\
Singapore \\
\texttt{h2030008@nushigh.edu.sg}}
\and
\IEEEauthorblockN{Prannaya Gupta}
\IEEEauthorblockA{Department of Arts and Humanities \\
NUS High School of Math and Science\\
Singapore \\
\texttt{h1810124@nushigh.edu.sg}}
\and
\IEEEauthorblockN{Marcus Ryan Loh}
\IEEEauthorblockA{Department of Arts and Humanities \\
NUS High School of Math and Science\\
Singapore \\
\texttt{h1810099@nushigh.edu.sg}}
}
%
\maketitle             % typeset the header of the contribution
%

\begin{abstract}
Insert abstract here
\end{abstract}

\begin{IEEEkeywords}
white supremacy, anti-black sentiment, Spider-Man, Peter Parker, Miles Morales, comic books, superheroes
 \end{IEEEkeywords}


\section{Introduction}

Spider-Man's name is ubiquitous in modern society, the unsung superhero that remains a staple of the modern comic book. With this popularity comes the necessity of adaptation. Above corporate greed, fans of the character crave for adaptations that are true to the source material, while offering something new. 

Largely, the novelty is in the medium. The 1967 \textit{Spider-Man} cartoon show came around largely because of the newly-introduced medium of animation and the opportunity to bring the character from the comic books alive. A show now most famous for the catchy theme song and the many, many memes created, the show was the beginning of Spider-Man's history of visual adaptation.

Soon after, more shows and movies followed. Spider-Man started appearing in short live-action segments on the \textit{Electric Company}, and another two short-lived live-action television shows (one in English and one in Japanese) came out in 1977 and 1978. The former birthed three feature films that have since fallen into the abyss.

On the animation front, Spider-Man remained a front-runner on the Saturday-Morning Cartoon Scene, with shows such as \textit{Spider-Man} (1981-1983), \textit{Spider-Man and his Amazing Friends} (1981-1982), the beloved \textit{Spider-Man: The Animated Series} (1994-1998) and more following.

In general, these adaptations have remained incredibly respectful to the source material, with the tragic character of Peter Parker being the sole Spider-Man. This was in line with the comics, where Peter Parker remained the one and only Spider-Man.

But since the 2000s, this trend has completely changed. Even dating as far back as November 1976, the character of Spider-Woman was the first step in several attempts to provide a diverse set of perspectives to 

\bibliographystyle{IEEEtran}
\bibliography{bib}
\end{document}
